\documentclass[12pt]{article}

\usepackage[T1]{fontenc}
\usepackage[latin1]{inputenc}
\usepackage[italian]{babel}
\usepackage{setspace}
\usepackage{graphicx}
\usepackage{makeidx}

\makeindex

\renewcommand{\ttdefault}{pcr}
\newcommand{\kw}[1]{\textbf{#1}}
\newcommand{\comment}[1]{\textit{#1}}
\newcommand{\Line}[0]{\rule{0cm}{0cm}\\\hrule\rule{0cm}{0cm}}

\addtolength{\parskip}{0.2\baselineskip}

\begin{document}

\frenchspacing

\newpage

\noindent
\textsc{\large{\textbf{Studio di euristiche per il miglioramento di\\algoritmi di ranking per il World--Wide Web}}}\\
{\footnotesize Riassunto della tesi di Laurea in Informatica di Marco Olivo, matr. n. 592150\\
Relatore: Dr. Sebastiano Vigna\\
Correlatori: Dr. Paolo Boldi, Dr. Massimo Santini\\
\textsc{Anno Accademico 2002-2003}}
\Line

%\setstretch{1.5}

Uno dei problemi aperti dell'informatica e dell'\textit{information
retrieval} pi� affascinanti degli ultimi anni � quello di rispondere
alle interrogazioni poste da esseri umani ai motori di ricerca.

I motori di ricerca si distinguono dalle basi di dati e dai sistemi di
\textit{information retrieval} tradizionali per diverse ragioni: il
World--Wide Web (web) � immenso\footnote{Si calcola che, compressa, la
porzione di web ad oggi raggiungibile occupi circa 50 terabyte, per un
totale di oltre tre miliardi di pagine.}, raddoppia di dimensione ogni
dodici mesi e si aggiorna con una frequenza impredicibile; inoltre i
dati non sono strutturati e sono altamente eterogenei nel contenuto e
nella qualit�.

Il recupero di tali quantit� di dati � reso difficile non soltanto
dalla loro mole, ma anche da una serie di fattori legati
all'affidabilit� dei server su cui tali dati vengono ospitati e dalle
reti attraversate per raggiungerli.

La studio dei motori di ricerca � relativamente recente, e le
pubblicazioni in letteratura su questo argomento, per quanto esso sia
molto dibattuto, tendono ad essere molto poche e soprattutto molto
poco approfondite; tra i motivi che spingono i ricercatori che se ne
occupano a non divulgare tutti i dettagli del loro lavoro vi sono
forti interessi economici legati all'uso commerciale dei motori di
ricerca.

\bigskip
Lo scopo di questa tesi � stata la costruzione di un motore di ricerca
scalabile, preciso e soprattutto flessibile e che potesse competere
con i motori di ricerca commerciali --- se non per numero di pagine
trattate, almeno per qualit� dei risultati.  Per raggiungere questi
obbiettivi si � partiti da una implementazione accurata dei migliori
algoritmi noti in letteratura, a cui � seguita una fase di affinamento
basata da un lato su considerazioni di tipo ingegneristico e
dall'altro da un attento confronto dei risultati ottentuti con quelli
restituiti dai motori di ricerca commerciali in risposta a varie
interrogazioni.

Lo schema seguito da un motore di ricerca nel momento in cui un utente
inserisce un'interrogazione �, in prima approssimazione, il seguente.
Anzitutto il motore opera una scrematura delle pagine che soddisfano
l'interrogazione sottopostagli, in maniera tale da eliminare subito un
numero significativo di pagine che probabilmente non interessano
all'utente; ciononostante, a causa delle dimensioni del web, il numero
di pagine rimanenti � in generale molto elevato. Pertanto,
l'ordinamento relativo di tali pagine � forse la cosa pi� importante
che un motore di ricerca si deve occupare di fornire, in quanto �
proprio tale ordinamento a dare all'utente la sensazione di aver
trovato quel che stava cercando.

Una tecnica possibile, ad esempio, � quella di preferire le pagine a
cui si riferiscono molte altre pagine, oppure quelle in cui i termini
dell'interrogazione compaiono in posizioni ravvicinate. Altre
tecniche, pi� sofisticate, si basano sulla struttura di
interconnessione tra le pagine del web.

\bigskip
Nei primi capitoli vengono anzitutto presentati ed analizzati alcuni
algoritmi noti proposti da tempo in letteratura e che si reputa
vengano utilizzati dai pi� famosi motori di ricerca commerciali per
decidere l'ordine di presentazione dei risultati.  Alcuni di questi
algoritmi sono stati utilizzati nel motore di ricerca che costituisce
lo scopo di questa tesi.

Nei capitoli successivi si analizzano alcune euristiche tese anzitutto
a migliorare la qualit� dei risultati e che cercano di fornire
risposte pi� mirate e precise alle interrogazioni sottoposte al motore
di ricerca, in maniera da emulare e, per quanto possibile, migliorare
lo stato attuale dell'arte, rappresentato in questo caso dai motori di
ricerca commerciali; in seconda battuta, vengono presentate altre
euristiche volte a diminuire i tempi di risposta alle interrogazioni,
in maniera da rendere realistico l'utilizzo del motore di ricerca
sviluppato.

Infine, � stata studiata ed implementata una tecnica per aggregare i
risultati dei vari algoritmi tra di loro in maniera efficiente ed in
modo altamente parametrico, cos� da rendere possibile la
sperimentazione dell'impatto dei vari algoritmi sulla qualit� della
risposta.

\end{document}

%%% Local Variables: 
%%% mode: latex
%%% TeX-master: t
%%% End: 
